%@AUTHOR: Cardel
%Configuracion del documento

\documentclass{beamer}
\usetheme{Rochester}
\usepackage{graphicx}
\usepackage[utf8]{inputenc}
\usepackage[spanish]{babel}
\usepackage{ragged2e}
\usepackage{colortbl}
\usepackage{color}
\definecolor{naranja}{rgb}{1,0.5,0} % valores de las componentes roja, verde y azul (RGB)
\definecolor{rojo}{rgb}{1,0,0}
\definecolor{SteelBlue}{rgb}{0.3,0.5,0.7}
\usepackage{listings}
\usepackage{listingsutf8}
\usepackage{float}
\usepackage{amsmath, amsthm, amssymb}
\lstset{ %
  basicstyle=\scriptsize,           % the size of the fonts that are used for the code
  numbers=none,
  numberstyle=\footnotesize,          % the size of the fonts that are used for the line-numbers
  numbersep=4pt,                  % how far the line-numbers are from the code
  backgroundcolor=\color{white},      % choose the background color. You must add \usepackage{color}
  breaklines=true,                % sets automatic line breaking
  breakatwhitespace=false,        % sets if automatic breaks should only happen at whitespace
  title=\lstname,                   % show the filename of files included with \lstinputlisting;{}
  extendedchars=false,
  inputencoding=utf8, 
}

\author{Carlos Andr\'es Delgado S.} 
\title{Arquitectura de computadores I}
\subtitle{Presentación del curso \\ carlos.andres.delgado@correounivalle.edu.co}
\institute{Facultad de Ingeniería. Universidad del Valle}
%Transparencia
\setbeamercovered{transparent}

%LOGO Univalle
\pgfdeclareimage[height=1.4cm]{logo}{imagenes/univalle}
\logo{\pgfuseimage{logo}}

%Para que en cada seccion aparezca la tabla de contenido
\AtBeginSection[]{
	\begin{frame}
	\frametitle{Contenido}
	\tableofcontents[currentsection]
\end{frame}
}

\date{Agosto de 2017}
\newcommand{\grad}{\hspace{-2mm}$\phantom{a}^{\circ}$}
\begin{document}

	\begin{frame}
		\titlepage	 		
	\end{frame}
	\begin{frame}
 		\frametitle{Contenido}
		\tableofcontents
	\end{frame}
	

	\section{Reglas}
	\begin{frame}
		\frametitle{El curso}
		\begin{itemize}
			\item Toda comunicación del docente será por el campus virtual. Recomiendo enlace el correo institucional a su cuenta, para que le lleguen los mensajes.
			\item Existe una bitácora en el campus virtual, que puede consultar para saber en que tema va el curso.
			\item El contacto del estudiante con el docente, será a través de correo electrónico.
			\item Me pueden buscar en el tiempo que esté en la sede, sin embargo, no les garantizo los pueda atender.
			\item Si considera que el correo no es lo ideal para su pregunta, podemos organizar una videollamada.
		\end{itemize}
	\end{frame}	
	
	\begin{frame}
		\frametitle{Sobre el correo electrónico}
		\begin{itemize}
			\item El tiempo de respuesta depende de la disponibilidad del docente, se hará lo más rápido posible, pero puede tardar en algunos casos, como los fines de semana o en horas de la noche.
			\item Al escribirme, es importante me indiquen de que curso son, para darle la asesoría apropiada.
			\item Se atenderán dudas de asignaciones (entrega, quices, parciales, etc) hasta 48 horas antes de su realización
			\item Comprendo que desean los resultados de las evaluaciones lo más rápido posible y me comprometo a ello, pero no contestaré correos presionando por estas.
			\item Pueden preguntar lo que quieran. Sin embargo, los invito que antes de preguntar, se intente solucionar el problema, para aprovechar la asesoría docente al máximo.
		\end{itemize}
	\end{frame}	

	\begin{frame}
		\frametitle{Sobre las entregas}
		\begin{itemize}
			\item Las entregas se realizarán por el campus virtual. En el caso de los quices será en la clase.
			\item El enlace del campus virtual tiene una hora de cierre fija. Sin embargo, puede entregar un poco más tarde (el enlace lo permitirá hasta 2 horas después del cierre) con una penalización de acuerdo al tiempo de entrega tardía.
			\item El curso tiene acceso de invitados (entrada sin sesión), por lo que debe tener cuidado no se le cierre la sesión.
			\item Si el campus falla, esperar a que se arregle para enviar su entrega. Avisarme por correo de la falla.
			\item En ningún caso, aceptaré entregas por correo electrónico.
		\end{itemize}
	\end{frame}	

	\section{Conocimientos previos}	

	\begin{frame}
		\frametitle{Conocimientos previos}
		\begin{block}{General}
			\begin{itemize}
				\item Manejo básico de un computador
				\item Entendimiento básico de las partes de un computador
			\end{itemize}
			Si tiene falencias en esto, no se preocupe. En esta etapa de su carrera, los puede asimilar rápidamente.
		\end{block}
	\end{frame}	
		

	
\begin{frame}
	\frametitle{¿Preguntas?}
	\centering
	Próximo tema: \\Introducción e historia del computador
\end{frame}					
			

\end{document}