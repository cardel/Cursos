\documentclass[twocolumn]{article}
\usepackage{graphicx}
\usepackage[utf8]{inputenc}
\usepackage[spanish]{babel}
\usepackage{ragged2e}
\usepackage{colortbl}
\usepackage{color}
\usepackage{float}

\definecolor{naranja}{rgb}{1,0.5,0} % valores de las componentes roja, verde y azul (RGB)
\definecolor{rojo}{rgb}{1,0,0}
\definecolor{SteelBlue}{rgb}{0.3,0.5,0.7}
\definecolor{violet}{rgb}{0.5,0,1}
\usepackage{listings}
\usepackage{listingsutf8}
\usepackage{amsmath, amsthm, amssymb}
\usepackage{textcomp}

\lstset{ %
  basicstyle=\scriptsize,           % the size of the fonts that are used for the code
  numbers=left,
  numberstyle=\footnotesize,          % the size of the fonts that are used for the line-numbers
  numbersep=4pt,                  % how far the line-numbers are from the code
  backgroundcolor=\color{white},      % choose the background color. You must add \usepackage{color}
  breaklines=true,                % sets automatic line breaking
  breakatwhitespace=false,        % sets if automatic breaks should only happen at whitespace
  extendedchars=true,
  inputencoding=utf8, 
    tabsize=8,
    aboveskip=5pt,
    upquote=true,
    showstringspaces=false,
    frame=single,
    showtabs=false,
    showspaces=false,
    showstringspaces=false,
    keywordstyle=\color{blue}\ttfamily\bfseries,
    stringstyle=\color{red}\ttfamily,
    commentstyle=\color{violet}\ttfamily,
    morecomment=[l][\color{magenta}]{\#},  
    literate={á}{{\'a}}1 {é}{{\'e}}1 {í}{{\'i}}1 {ó}{{\'o}}1 {ú}{{\'u}}1 {'}{{'}}1 {ñ}{{\~n}}1 {&}{{\&}}1 , 
}
\usepackage{anysize} 
\marginsize{1cm}{1cm}{1cm}{2cm} 
\let\olditemize\itemize
\let\oldenumerate\enumerate
\def\itemize{\olditemize\itemsep=0pt } %%Reducir espacio itemize
\def\enumerate{\oldenumerate\itemsep=0pt } %%Reducir espacio itemize
\setlength{\tabcolsep}{12pt}
\setlength{\columnsep}{20pt}
\title{\vspace{-2cm} \includegraphics[scale=0.2]{univalle.jpg} \\Segundo Taller\\ FUNDAMENTOS DE ANALISIS Y DISEÑO DE ALGORITMOS \vspace{-0.5cm}}
\author{Carlos Andres Delgado S, Ing \footnote{ carlos.andres.delgado@correounivalle.edu.co }}
\date{\vspace{-0.2cm}Marzo 2017}


\newcommand{\raya}{\underline{\hspace{3cm}}}
\newcommand{\grad}{\hspace{-2mm}$\phantom{a}^{\circ}$}
\begin{document}
\maketitle

 
\section{Análisis de algoritmos}

Dar la complejidad en términos de $O(f(n))$ y analizar el mejor caso, caso promedio y peor caso. Suponga que $n>0$. En el caso de los algoritmos iterativos, analizar: estado, estado inicial, transformación de estados, estado final e invariante de ciclo.

Los siguientes algoritmos iterativos :

\begin{lstlisting}
//a: arreglo de tamaño n
//n: tamaño del arreglo a
//Primera posición arreglos es 0
algoritmo1(a, n)
   i = 0
   b = 0

   while(i < n)   
          j = 0         
          while(j < n)
          	s = a[j]
          	b = b + b*i
            while(a[i] < a[j])
               t = a[j]
               a[j] = a[i]
               a[i] = t
            end
            j++
          end
          i++
   end
   
   print b
end
\end{lstlisting}

\begin{lstlisting}
algoritmo2(n)
   b = 0
   c = 1
   j = n
   while j >= 2
       a = b
       b = c
       c = b + a
       j--
   end   
   print c 
\end{lstlisting}

Los siguientes algoritmos recursivos.

\begin{lstlisting}
algoritmo3(n)
    if n == 0
        return 1
    else
        return algoritmo3(n-1) + algoritmo3(n-1) + algoritmo3(n-1)
\end{lstlisting}
\newpage
\begin{lstlisting}
//a es un arreglo de tamaño n
algoritmo4(a,n)
   if n == 0 return a
   else
      n--
      B = ordenar(a,n)
      return algoritmo(B,n-1)
   end
end

ordenar(a,n)
     i = 1
     x = a[0]
     while i < n
        if(a[i]>x)
          t = a[i-1]
          a[i-1] = a[i]
          a[i] = t 
          x = a[i]
        
        i++
     end
end
\end{lstlisting}

\section{Ecuaciones de recurrencia}

Para todos los casos suponga que $T(1) = O(1)$ \\\\

Resolver utilizando método de árboles, basta con dejar expresada la sumatoria en los casos que no la pueda solucionar.

\begin{itemize}
	\item $T(n) = 4T(\frac{n}{4}) + n + 2$
	\item $T(n) = 2T(\frac{n}{4}) + n^2$
	\item $T(n) = 2T(\frac{n}{2}) + log(n)$ 
\end{itemize}

Resolver utilizando método de sustitución.

\begin{itemize}
	\item $T(n) = T(n) = 2T(n-1) + 4$, es $O(2^n)$
	\item $T(n) = 2T(\frac{n}{2}) + 3n$ es $O(nlog(n))$
\end{itemize}


Resolver utilizando método del maestro


\begin{itemize}
	\item $T(n) = 4T(\frac{n}{4}) + n + 2$
	\item $T(n) = 4T(\frac{n}{2}) + 2n$
	\item $T(n) = 2T(\frac{n}{4}) + n^2$
\end{itemize}



\end{document}