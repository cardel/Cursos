\documentclass[10pt,twocolumn]{article}
\usepackage[utf8]{inputenc} %Para configuración de caracteres
\usepackage[spanish]{babel} %Para configuración de idioma
\usepackage{graphicx}
\usepackage{xcolor}
\usepackage{amsfonts}
\usepackage{listings}
\usepackage{textcomp}
\usepackage{amsmath}

\lstset{ %
  language=C++,
  basicstyle=\scriptsize,           % the size of the fonts that are used for the code
  numbers=none,
  numberstyle=\footnotesize,          % the size of the fonts that are used for the line-numbers
  numbersep=4pt,                  % how far the line-numbers are from the code
  backgroundcolor=\color{white},      % choose the background color. You must add \usepackage{color}
  breaklines=true,                % sets automatic line breaking
  breakatwhitespace=false,        % sets if automatic breaks should only happen at whitespace
  extendedchars=true,
  inputencoding=utf8, 
  tabsize=4,
  aboveskip=5pt,
  upquote=true,
  showstringspaces=false,
  frame=single,
  showtabs=false,
  showspaces=false,
  showstringspaces=false,
  keywordstyle=\color{blue}\ttfamily\bfseries,
  stringstyle=\color{red}\ttfamily,
  commentstyle=\color{violet}\ttfamily,
  morecomment=[l][\color{magenta}]{\#}, 
  escapechar=¡,
  literate={á}{{\'a}}1 {é}{{\'e}}1 {í}{{\'i}}1 {ó}{{\'o}}1 {ú}{{\'u}}1 {'}{{'}}1, 
}
\lstset{emph={print
  },emphstyle={\color{blue}\ttfamily\bfseries}%
}
%\lstset{emph={%  
%    lambda, define, empty, number, cadr, caddr, if, else, iszero, define-datatype, pair, symbol, eqv,car, cdr, cases, sllgen, %skip, in, then%
%    },emphstyle={\color{blue}\ttfamily\bfseries}%
%}%

\usepackage{anysize} 
\marginsize{2cm}{2cm}{1cm}{2cm} 

\title{\vspace{-2.5cm} \includegraphics[scale=0.15]{univalle.jpg} \\Primer examen opcional \\ Fundamentos de análisis y diseño de algoritmos \\ \vspace{-0.5cm}}
\author{Carlos Andres Delgado S, Ing \footnote{ carlos.andres.delgado@correounivalle.edu.co }}
\date{\vspace{-0.2cm}01 de Julio 2017}


\newcommand{\raya}{\underline{\hspace{3cm}}}
\newcommand{\grad}{\hspace{-2mm}$\phantom{a}^{\circ}$}
\begin{document}
\maketitle
\vspace{-0.5cm}
{\bf Nombre:\underline{\hspace{6cm}}}
\hfill
{\bf Código:\raya}


\section{Ecuaciones de recurrencia \small{[10 puntos]}} 

 Utilizando el método de árboles, solucione la siguiente ecuación de recurrencia
	\begin{align*}
	      T(n) = 4T(\frac{n}{3}) + n, T(1) = O(n^3)
	\end{align*}	


\section{Divide y vencerás \small{[20 puntos]}} 

Diseñe un algoritmo mediante la técnica de divide y vencerás para encontrar el mayor y el segundo mayor elemento en una lista. Indique la complejidad de este algoritmo.

\section{Ordenamiento \small{[10 puntos]}} 

Indique un algoritmo para ordenar números  entre 1 y $n^3$ que se encuentran distribuidos uniformemente en tiempo $O(n)$.

\section{Computación iterativa \small{[60 puntos]}} 

\begin{enumerate}
	\item (20 puntos) Indique la complejidad computacional en términos de $O(f(n))$ con el $f(n)$ más pequeño posible, de los siguientes algoritmos:

\begin{lstlisting}
//Para n > 2
for(int i=2; i<n; i=i*i){
	for(int j=0; j<n*n; j++){
		//...	
	}
}
\end{lstlisting} 
\begin{lstlisting}
//Para n >= 0
for(int i=2n; i>=0; i--){
	for(int j=0; j<4n; j=j+2){
		//...	
	}
}
\end{lstlisting} 	
\begin{lstlisting}
//Para n >= 0
for(int i=0; i<=n; i++){
	for(int j=i; j<=8n; j=j++){
		//...	
	}
}
\end{lstlisting} 	

	\item (20 puntos) Para el siguiente algoritmo indique:
	\begin{itemize}
		\item Forma de estado, estado inicial
		\item Transformación de estados y estado final
		\item Invariante de ciclo
	\end{itemize}
\begin{lstlisting}[language=Java]
Algoritmo(int N)
{
	int i,res;
	i = -6;
	res = 1;	
	
	while(i<=N*N){
		
		res = res + 3*i;
		
		for(int j=0; j<=2N; j+=2)
		{
			res = 2*res+i;
		}
		i++;
	}
}
\end{lstlisting} 
	\item (20 puntos)Para el siguiente algoritmo, el cual recibe un arreglo A y su tamaño n.

\begin{lstlisting}[language=Java]
Algoritmo2(int A[], int n)
{
	int res = 0;
	for(int i=0; i<n; i++){
		if(A[i] % 2 == 0){
			for(int j=i; j<=n*n; j++)}{
				res = res+A[i];	
			}
		}
		else{
			res = res+2+A[i];	

		}
	}
}
\end{lstlisting} 
En términos de $O(f(n))$ con el $f(n)$ más pequeño posible, indique la complejidad del algoritmo para:
\begin{itemize}
	\item Mejor caso
	\item Peor caso
	\item Caso promedio
\end{itemize}
	\end{enumerate}
\begin{center}
	\huge{¡Exitos!}
\end{center}	
	
\end{document}