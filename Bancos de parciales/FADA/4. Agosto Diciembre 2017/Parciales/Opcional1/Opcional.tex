\documentclass[9pt,twocolumn]{article}
\usepackage[utf8]{inputenc} %Para configuración de caracteres
\usepackage[spanish]{babel} %Para configuración de idioma
\usepackage{graphicx}
\usepackage{xcolor}
\usepackage{amsfonts}
\usepackage{listings}
\usepackage{textcomp}
\usepackage{amsmath}

\lstset{ %
  language=C++,
  basicstyle=\scriptsize,           % the size of the fonts that are used for the code
  numbers=none,
  numberstyle=\footnotesize,          % the size of the fonts that are used for the line-numbers
  numbersep=4pt,                  % how far the line-numbers are from the code
  backgroundcolor=\color{white},      % choose the background color. You must add \usepackage{color}
  breaklines=true,                % sets automatic line breaking
  breakatwhitespace=false,        % sets if automatic breaks should only happen at whitespace
  extendedchars=true,
  inputencoding=utf8, 
  tabsize=4,
  aboveskip=5pt,
  upquote=true,
  showstringspaces=false,
  frame=single,
  showtabs=false,
  showspaces=false,
  showstringspaces=false,
  keywordstyle=\color{blue}\ttfamily\bfseries,
  stringstyle=\color{red}\ttfamily,
  commentstyle=\color{violet}\ttfamily,
  morecomment=[l][\color{magenta}]{\#}, 
  escapechar=¡,
  literate={á}{{\'a}}1 {é}{{\'e}}1 {í}{{\'i}}1 {ó}{{\'o}}1 {ú}{{\'u}}1 {'}{{'}}1, 
}
\lstset{emph={print
  },emphstyle={\color{blue}\ttfamily\bfseries}%
}
\lstset{emph={%  
    end%
    },emphstyle={\color{blue}\ttfamily\bfseries}%
}%

\usepackage{anysize} 
\marginsize{1cm}{2cm}{1cm}{2cm} 

\title{\vspace{-2.5cm} \includegraphics[scale=0.15]{univalle.jpg} \\Primer examen opcional \\ Fundamentos de análisis y diseño de algoritmos \\ \vspace{-0.5cm}}
\author{Carlos Andres Delgado S, Ing \footnote{ carlos.andres.delgado@correounivalle.edu.co }}
\date{\vspace{-0.2cm}15 de Noviembre 2017}


\newcommand{\raya}{\underline{\hspace{3cm}}}
\newcommand{\grad}{\hspace{-2mm}$\phantom{a}^{\circ}$}
\begin{document}
\maketitle
\vspace{-0.5cm}
%{\bf Nombre:\underline{\hspace{6cm}}}
%\hfill
%{\bf Código:\raya}
%\\
\textbf{Importante:} Se debe escribir el procedimiento realizado en cada punto, con sólo presentar la respuesta, el punto no será válido.



\section{Divide y vencerás \small{[35 puntos]}} 


Dado un vector de puntos en un sistema de dos dimensiones, Encontrar el punto cuya distancia al punto (0,0) sea la menor. Ejemplo $\{(0,5),(1,2),(3,6),(2,7)\}$ Si calculamos la distancia de los elementos al puntos $(0,0)$ tenemos: $\{5,2.23,6.7,7.2\}$ ordenando de acuerdo a la distancia tenemos $\{(1,2),(0,5),(3,6),(2,7)\}$ Entonces el punto más cercano es $(1,2)$

\begin{itemize}
	\item (5 puntos) Indique la solución ingenua al problema.
	\item (20 puntos) Estrategia de dividir, vencer (solución trivial), y combinar. Muestre un ejemplo de solución del problema con un vector de tamaño 6.
	\item (10 puntos) Calculo de la complejidad computacional de la solución utilizando divide y vencerás. Explique cómo realiza este proceso. Compare este resultado contra la complejidad de la solución ingenua y una solución que consiste en ordenar de acuerdo a la distancias y escoger el primero. ¿Cual de las tres soluciones escoge y porqué?
\end{itemize}

\section{Computación iterativa \small{[30 puntos]}} 

\begin{lstlisting}[numbers=left]
//Para n > 0
algoritmo(n)
	i = 0
	res = 0
	
	while(i<n)
		j = i
		
		while(j<2n)
			res += 3
			j++
		end		
		
		res += 5
		i++
	end
end
\end{lstlisting} 

\begin{enumerate}
	\item \textbf{\textbf{(15 puntos)}} Complejidad computacional. Muestre cuantas veces se ejecuta cada línea en términos de $n$
	\item \textbf{\textbf{(20 puntos)}} Invariante de ciclo para ciclos interno y externo. Indique cómo es la forma de los estados inicial, final y la transformación de estados. Demuestre la invariante de ciclo con respecto a estos. 
\end{enumerate}
\section{Crecimiento de funciones \small{[35 puntos]}} 

Dado que $f(n)$ y $g(n)$ son funciones positivas y crecientes, para $n\geq 0$. Demostrar o refutar las siguientes conjeturas para todo  $f(n)$ y $g(n)$

\begin{itemize}
	\item $f(n) = O(g(n))$ entonces $g(n)=O(f(n))$ 
	\item $f(n) = O(f(\frac{n}{2}))$ 
	\item $f(n) + o(f(n)) = \Theta(f(n))$
	\item $f(n) = \Theta((f(n))^2)$
	\item $f(n) = O(g(n))$ implica $g(n) = \Omega(f(n))$
\end{itemize}


\section*{Ayudas}
\begin{tabular}{ccc}
 $\sum \limits_{k=1}^n c = cn$ & $\sum \limits_{k=1}^n k = \frac{n(n+1)}{2}$ & $\sum \limits_{k=1}^n k^2 = \frac{n(n+1)(2n+1)}{6}$ \\
  \multicolumn{3}{c}{$\sum \limits_{k=0}^n ar^k = \frac{ar^{(n+1)}-a}{r-1}$ Si $r \neq 1$ } \\
 \multicolumn{3}{c}{$\sum \limits_{k=0}^n ar^k = (n+1)a$ Si $r = 1$ } 
\end{tabular}

\end{document}